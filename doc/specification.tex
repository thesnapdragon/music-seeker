\section{Feladatkiírás}
\label{sec:feladatkiiras}

A felhasználó eszközein (asztali PC vagy mobileszköz) tárolt zenéi alapján a rendszer megadja, hogy melyik zeneszolgáltatás katalógusában (Spotify, Deezer, Google Music,iTunes) található meg a legtöbb a tárolt zenék közül. Ehhez a különböző zeneszolgáltatások API-ját használja fel.

% section feladatkiiras (end)

\section{A fejlesztői csapat}
\label{sec:afejlesztoicsapat}

A csapat tagjai:

\begin{table}[htb]
\begin{center}
\begin{tabular}{|l|l|l|}
\hline
\textbf{Csapattag neve} & \textbf{Neptun-kód} & \textbf{E-mail cím} \\ \hline
Fekete Norbert Zoltán & CO0DA1 & feno26@gmail.com \\ \hline
Unicsovics Milán György & M9GNTV & u.milan@gmail.com \\ \hline
\end{tabular}
\end{center}
\label{tab:acsapattagjai}
\caption{A csapat tagjai}
\end{table}

A csapatban dedikált szerepek kiosztását a csapat kis mérete miatt nem tartottuk fontosnak.

% section afejlesztoicsapat (end)

\section{Részletes feladatleírás}
\label{sec:reszletesfeladatkiiras}

A projekt során célunk egy olyan alkalmazás készítése, amely segít a felhasználónak eligazodni a manapság egyre inkább elterjedő internetes zenei szolgáltatások világában. Ezt a felhasználó meglévő zenei gyűjteményének letapogatásával, majd annak a felhő alapú szolgáltatók készleteivel történő összevetésével éri el.

Az alapvető keresési egység a zenei album lesz. Az elemzési folyamat végeztével a felhasználó egy statisztikát kap, melyből kiolvashatja, mely zeneszolgáltatás gyűjteményével a legnagyobb az átfedés - vagyis mely szolgáltatónál találhat a legkönnyebben a saját stílusának megfelelő zenéket.

Emellett egyszerű, kulcsszavas keresésre is lesz lehetőség (meglévő zenefájlok nélküli kereséshez), amivel kényelmesen lehet majd egyszerre több szolgáltató készletét is lekérdezni.

A program első megközelítésben a \textit{Deezer}, a \textit{Spotify}, az \textit{iTunes} és a \textit{Last.Fm} szolgáltatásait fogja támogatni, de célunk egy általános architektúra kialakítása, mely később könnyedén bővíthető további szolgáltatásokkal (pl.: \textit{Google Play Music}).

% section reszletesfeladatkiiras (end)

\section{Technikai paraméterek}
\label{sec:technikaiparameterek}

A definiált alkalmazást Python platformra készítjük el annak érdekében, hogy több operációs rendszeren (Windows, Linux, Mac OS) is lehessen használni.  Szükség lesz ezen kívül néhány Python-os könyvtárhoz, ezeket a Python csomagkezelőjével (\textit{pip}) egyszerűen lehet majd telepíteni.

A program felhasználói felületét \href{http://python-gtk-3-tutorial.readthedocs.org/en/latest/}{Python GTK+ 3} segítségével fogjuk elkészíteni, a GUI-t magát deklaratív módon \href{https://glade.gnome.org/}{Glade} segítségével állítjuk elő.

Külső függőségeink közé tartoznak a zeneszolgáltatások API-jai is:

\begin{itemize}
	\item \href{https://developer.spotify.com/web-api/}{Spotify}
    \item \href{http://developers.deezer.com/api/}{Deezer}
    \item \href{https://www.apple.com/itunes/affiliates/resources/documentation/itunes-store-web-service-search-api.html}{iTunes}
    \item \href{http://www.last.fm/api}{Last.fm}
    \item \href{http://unofficial-google-music-api.readthedocs.org/en/latest/}{(Google Music)}
\end{itemize}

A fent felsorolt külső függőségek sebességének függvényében, lehet, hogy szükség lesz valamilyen szerver oldali komponens fejlesztésére is, amely egy fajta cache-ként szolgálva gyorsíthatja a program működését. A szerver elérhetősége viszont a program funkcionalitását nem befolyásolja.

% section technikaiparameterek (end)

\section{Szótár}
\label{sec:szotar}

% section szotar (end)

\begin{description}
    \item[Zeneszolgáltatás] Kereskedelmi streamelő alkalmazás, melyben a fehasználó zenéket hallgathat illetve vásárolhat.
    \item[Zeneállomány] A zeneszolgáltatás vagy a felhasználó által birtokolt zenealbumok összessége.
    \item[Zenei katalógus] A zeneszolgáltatás kereshető zeneállománya.
    \item[Kiértékelési statisztika] A felhasználó által megtekinthető grafikus szemléltető eszköz, mely arra szolgál, hogy az egyes zeneszolgáltatásoknál, mely zenealbumok érhetőek el.
    \item[Zeneállomány felderítése] A felhasználó által megadott helyen az elérhető zenealbumok és azok adatainak összegyűjtése.
    \item[Keresés zenékre] Minden támogatott zeneszolgáltató átvizsgálása, hogy egy bizonyos zenealbum elérhető-e az adott platformon.
\end{description}

\section{Essential use-case-ek}
\label{sec:usecaseek}

\subsection{Use-case diagram}
\label{sub:ucdiagram}

\begin{figure}[htp]
\centering
\includegraphics[scale=0.8]{img/01_UseCases.png}
\caption{A MusicSeeker use case diagramja}
\label{fig:01_UseCases}
\end{figure}


% subsection ucdiagram (end)

% section usecaseek (end)
